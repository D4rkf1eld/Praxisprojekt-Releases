\subsection{Historie}
    1888 gründete Paul Heinrich Trützschler im sächsichen Crimmitschau das Unternehmen.
    Die ersten Reißmaschinen, die zum recyclen von Textilien genutzt werden und die ersten Krempelwölfe wurden 1900, im Beisein des Teilhabers Bruno Gey, entwickelt und produziert. \cite{noauthor_trutzschler_nodate}
    Später folgten dann Baumwollreinigungsanlagen und der weltweit erste Chemiefaseröffner.

    1948, nach dem zweiten Weltkrieg und der Enteignung, wurde das Werk in Mönchengladbach-Odenkirchen von Hans und Hermann Trützschler, den Enkeln des Firmengründers, neu gegründet. \cite{noauthor_trutzschler_nodate-1}
    Durch die Gründung der Tochtergesellschaft American Trützschler Inc. 1969 in Charlotte (USA) und den Ausbau des Produktportfolios wurde der Grundstein der weltweiten Expansion gelegt.
    In der mittlerweile 4. Generation wird die Trützschler Group von den Mitgliedern und geschäftsführenden Gesellschaftern der Inhaberfamilien Trützschler und Schürrenkamp geführt. \cite{noauthor_familienunternehmen_nodate}