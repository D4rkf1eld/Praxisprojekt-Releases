\subsection{Besonderheiten}
    Bedeutsam ist die eingangs schon erwähnte hohe Eigenfertigungstiefe, die innerhalb der einzelnen Manufakturschritte der Maschinen und Anlagen erreicht wird.
    Das bedeutet, dass sowohl die elektrotechnischen Komponenten, wie z.~B. Steuergeräte oder Motortreiberstufen und die physisch-mechanischen Bestandteile, z.~B. die Gehäuse der Maschinen, allesamt \glqq im eigenen Hause\grqq\@ entwickelt und produziert worden sind.
    Diese hohe Eigenfertigungstiefe kann viele Vorteile haben.

    Generell spielen strategische Überlegungen, sowie qualitative Aspekte, d.h. die Lieferbereitschaft der externen Manufakteuren oder das Qualitätsniveau der eigenen Produktpalette, eine große Rolle, die bei der Wahl zur Eigenfertigung berücksichtigt werden müssen.
    Auch sind häufig marktstrategische Punkte relevant. \cite{erichsen_eigenfertigung_nodate}

    Die Branche des Textilmaschinenbaus ist nicht gesättigt, sodass ein Prestige für hochqualitative Maschinen und Anlagen aufgebaut werden kann.
    So bietet Trützschler, aufgrund der direkten Zugriffsmöglichkeit auf die einzelnen Bauelemente und gefertigten Leiterkarten, für die eigenen Produkte ein Ersatzteilgeschäft an und kann eine \notsure{30 jährige Garantie} aussprechen.
       
    % Durch die hohe Eigenfertigungstiefe kann das Know-How in der Elektronikfertigung gesichert werden.
    Durch die bewusste Eigenfertigung der Maschinen und Anlagen wird auch das Know-How zur Herstellung und zur genauen Funktionsweise dieser in der Elektronikfertigung gesichert.
    
    Das Know-How erlaubt, dass Entscheidungen bei der Bestückung auf Komponentenebene getroffen werden können, sodass die Leiterkarten hochspezialisiert und für ihren jeweiligen Anwendungsfall angepasst sind.
    Dies spart nicht nur Kosten, da die Kostenquantifizierung indirekt durch die benötigten Funktionalitäten und somit direkt über die benötigten Bauelemente stattfinden kann.
    Auch können die im Know-How enthaltenen Prozessoptimierungskenntnisse bei der Qualitätssicherung und der Fehlerbehebung eingesetzt werden, denn die Rückmeldungen des Test- und Produktionsbereiches können beispielsweise direkt zur Entwurfsabteilung rückgekoppelt werden.
    Im Falle eines Falles ist für den Gewinn nötiger Erkenntnisse von neuen Produkten der Bau eines Prototypes problemlos möglich und es ergeben sich, auch durch die enge räumliche und fertigungstechnische Zusammenarbeit des Entwicklungs- und Produktionsbereiches, sehr kurze Produktentwicklungszeiten. 

    Aufgrund dieser besonderen Punkte ist es nötig, dass Trützschler ihre Produkte vor Ort testet.

    % \annot{\notsure{Viel gemeinsames Wissen und hohe Herstellungsflexibilität notwendig, deswegen vertikal integriert?}}
    % Die Trützschler Group zeichnet sich durch eine hohe vertikale Integration der Manufakturschritte der Maschinen und Anlagen aus.
    % Das bedeutet, dass die elektrotechnischen Komponenten wie Steuergeräte, Motortreiberstufen, etc. und die physisch-mechanischen Bestandteile wie das Gehäuse(-design) allesamt aus einem Haus stammen.

    % \annot{Trützschler baut sehr viel selbst}

    %    \indent \indent \annot{Wärmekammern}
        
    %    \indent \indent \annot{Boards werden selbst \notsure{hergestellt?} designed und bestückt}

    %        \indent \indent \indent \annot{Warum eigentlich?}

    %        \indent \indent \indent \annot{Daher sind automatisierte/systematische Testmethoden für die Qualitätssicherung nötig}

    %    \indent \indent \annot{\notsure{Gehäuse werden selbst gemacht?}}