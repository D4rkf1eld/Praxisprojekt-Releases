\subsection{Das Geschäftsfeld}
    Das Geschäftsfeld umfasst vier Gebiete, wobei allesamt auf die Entwicklung und Produktion von Anlagen zur Faserverarbeitung spezialisiert sind: \cite{noauthor_familienunternehmen_nodate}

    \begin{enumerate}
        \item \textbf{Spinning} (Mönchengladbach)
        \item \textbf{Nonwovens} (Dülmen und Egelsbach)
        \item \textbf{Card Clothing} (Neubulach) 
        \item \textbf{Man-Made Fibers}
    \end{enumerate}

    Für die Spinnereivorbereitung von Baumwolle und Chemiefasern ist Trützschler \textbf{Spinning} Technologie- und Marktführer \cite{noauthor_trutzschler_nodate}.
    Hier werden Putzereimaschinen\footnote{Typische Prozesse einer Putzerei sind die Ballenöffnung, die Reinigung, die Fremdteilausscheidung und die Flockenmischung der Wolle zur besseren Weiterverarbeitung.} und Karden\footnote{Die losen Fasern der Wollflocken werden durch das Kadieren/Krempeln in eine Richtung \glqq gekämmt\grqq\@ und somit zu einem Vliessstoff ausgerichtet.}, sowie Strecken\footnote{Durch das gleichzeitige Strecken und Verdrehen der kadierten Fasern wird das Garn gesponnen \cite{tuchfabrik_von_2016}.} entwickelt und produziert \cite{noauthor_trutzschler_nodate-1}.

    Die \textbf{Nonwovens} (z.~Dt. Vliessstoff) Sparte produziert Maschinen und Anlagen für den gesamten Produktionsprozess (vom Öffnen bis zum Wickeln) von wasserstrahlverfestigten, luftdurchströmten und chemisch verfestigten Materialien \cite{noauthor_trutzschler_nodate-1}.

    Trützschler \textbf{Card Clothing} produziert Hochleistungsgarnituren\footnote{Eine Garnitur bezeichnet ein Metallkonstrukt aus vielen Metallbürsten, die für die Ausrichtung der unkadierten Wolle durch Walzen verantwortlich ist.} für Karden und Krempeln\footnotemark[2], die bei Trützschler \textbf{Spinning} und Trützschler \textbf{Nonwovens} zum Einsatz kommen \cite{noauthor_trutzschler_nodate-1}.

    % Spinnen ist das gleichmäßige Strecken und Verdrillen der Kadierten (gekämmten) Wolle. Die kadierte Wolle ist im industriellen Prozess der Flor (der Vliessstoff) der Krempelmaschine.

    Der Trützschler \textbf{Man-Made Fibers} Geschäftszweig stellt Anlagen zur Produktion technischer Garne \cite{noauthor_trutzschler_nodate}, Anlagen zur Produktion von Teppichgarnen (BCF-Garne) und auch Anlagen für den Chemiefaserbereich her \cite{noauthor_trutzschler_nodate-1}.