\subsubsection{Motivation}
    \label{subsubsection: Motivation}
    Die intensive Nutzung elektronischer Geräte im Alltag ist überhaupt erst durch Verbesserungen in der mikroelektronischen Schaltungstechnik möglich geworden \cite{grout_integrated_2006}.
    Gleichzeitig steigt der Bedarf an immer kleineren und kostengünstigeren Produkten mit ständig verbesserten und vereinfachten Funktionen für den Konsumenten \cite{berger_test-_2012}.
    Jedoch erwarten Nutzer elektronischer Produkte eine stets hohe Zuverlässigkeit, bei stetigen Produktverbesserungen der elektronischen Geräte.
    Gerade dann, wenn diese Produkte in anspruchsvollen sicherheits- und zeitkritischen Anwendungen integriert sind, wäre ein Fehlverhalten, oft hervorgerufen durch mangelnde Qualitätskontrolle, fatal, da im schlimmsten Fall Menschenleben gefährdet werden würden.
    Daher obliegt die Feststellung der Zuverlässigkeit als ein Aspekt, der die Qualität eines Produktes ausmacht, bei dem Hersteller des elektronischen Erzeugnisses.
    Dazu muss die Zuverlässigkeit in den einzelnen Produktionsphasen verifiziert werden können.
    Hierzu werden häufig automatisierte Systeme genutzt \ac{ate} und auch manuelle Menschen, weil blabla.. Die Systeme genügen folgenden Kriterien...
    \annot{Die Verifikation sollte allerdings bestimmten Kriterien genügen, damit...}
    Für zuverlässige elektronische Systeme stellen die Prüfverfahren, die oftmals von automatisierten und spezialisierten Prüfsystemen \notsure{(ein sog. \ac{ate})} durchgeführt werden, somit ein unerlässliches Mittel der Sicherheitsüberprüfung und Qualitätssicherung dar, wobei gleichzeitig der Kostenaspekt mit berücksichtigt werden muss.
    Deswegen gilt es zu überprüfen, wie die Testverfahren auf die folgenden Aspekte einwirken:

    \begin{itemize}
        \item \textbf{Qualitätssicherung}
        \item \textbf{Kostenminimierung}
        \item \textbf{Sicherheit}
    \end{itemize}

    Diese Auswirkungen werde ich in den folgenden Abschnitten näher untersuchen.

%----------------------------- Prototypisches -----------------------------%
    % Deprecated
        % Die Prüfverfahren greifen bei den folgenden Aspekten ein:

        % \begin{itemize}
        %    \item Qualitätssicherung
        %    \item Kostenminimierung
        %    \item Sicherheit
        % \end{itemize}

        % Die in Abschnitt \ref{subsection: Grundlegendes} angesprochenen zeitkritischen- und sicherheitskritischen Steuerungssysteme unterliegen oft hohen Sicherheitsauflagen.
        % So dürfen zum Beispiel im Betrieb der Steuerungssysteme keine Fehler auftreten.
        % Falls doch Fehler auftreten sollten, so muss das Fehlverhalten zumindest abschätzbar und regulierbar sein.
        % Daher erwarten Nutzer elektronischer Erzeugnisse eine hohe Zuverlässigkeit der verwendeten Komponenten.
%----------------------------- Prototypisches -----------------------------%