\subsection{Vorwort}
    Da ich beim Vorstellungsgespräch bei der Trützschler Group sehr von der Fertigungstiefe und der Komplexität der hier eigens entworfenen und bestückten Leiterkarten beeindruckt war, habe ich mich für das Praxisprojekt in der hauseigenen Elektronikfertigung entschieden.
    
    Faszinierend ist, dass die Komplexität der Leiterkarten durch verschiedene menschliche und automatisiert-maschinelle Herstellungsschritte erreicht wird.
    Von der Bestückung, bis hin zum vollständigen Systemtest, im Zusammenspiel mit weiteren Leiterkarten, werden hier in der Elektronikfertigung alle Prozesse selbst durchgeführt und zur anschließenden Verifikation dokumentiert.
    Aufgrund der Symbiose dieser Herstellungsschritte ist es möglich, eine große Stückzahl an komplexen Leiterkarten bei einer konstanten, hohen Qualität produzieren zu können.
    Damit bei diesen Stückzahlen die konstante und hohe Produktionsqualität dauerhaft gewährleistet werden kann, wird das Augenmerk von Beginn an besonders auf die korrekte Funktionsweise gelegt, die durch die verschiedenen Testverfahren gesichert wird.
    % Trützschler setzt hierzu zwei Multifunktionstester von SPEA und ein altes, obsoletes Gerät von Rohde \& Schwarz ein.
    
    Hierzu setzt Trützschler für den \ac{ict} den \glqq3030 Compact\grqq\@ von SPEA und die mittlerweile obsolete Testworkstation TSAS von Rohde \& Schwarz ein.
    Für den \ac{fpt} wird der \glqq4060\grqq\@ von SPEA verwendet.

    Zusätzlich hat mich begeistert, dass Trützschler auch seine eigenen Testsysteme entwickelt hat und diese in der Elektronikfertigung intensiv nutzt.
    Zum einen wäre da der Funktionstester TST-3 und zum anderen die Wärmekammern, die den Betrieb der Bauteile auf den Leiterkarten an den Spezifikationsgrenzen überprüfen.