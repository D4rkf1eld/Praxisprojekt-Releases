\subsection{Zielsetzung des Praxisprojektes}
    Ziel des Praxisprojektes ist, mehr über die systematische Elektronikherstellung im betrieblichen Sinn zu erlernen.
    Dazu möchte ich erfahren, wie die Leiterkarten im \glqq großen Stil\grqq\@ hergestellt werden und warum das Testen der Baugruppen unbedingt erforderlich ist.
    Zudem möchte ich mir Wissen über die möglichen und aktuellen Testverfahren aneignen, da ich bisher noch überhaupt gar keine \glqq Berührungen\grqq\@ damit hatte.
    Weiter möchte ich mich über die Messtechniken informieren, die das Testen überhaupt erst möglich machen.

    Da ich zur Lösung der Problemstellung in der noch folgenden Bachelorthesis eine umfangreiche Dokumentation der bei Trützschler verwendeten Tester zusätzlich benötige, werde ich in zwei dedizierten Kapiteln näher auf die Maschinen eingehen und diese dokumentieren.

    Dabei möchte ich im Vorfeld mögliche Migrationsstrategien anschneiden und die Gemeinsamkeiten und Unterschiede der Tester herausstellen.
    Zur Eingrenzung werde ich die Migrationsstrategien jedoch nur kurz beleuchten.
    
    % Aufgrund der immer weiter voranschreitenden Miniaturisierung von Baugruppen und Bauelementen wird die Fehlerdiagnose immer aufwändiger. 
    % Im Falle der \dbg{Akronym!} IC-Bausteine ist beispielsweise die Überprüfung der korrekten Kontaktierung oft garnicht möglich.
    % Somit stellen diese Testverfahren einen wichtigen und essentiellen Herstellungsschritt dar.
    % Daher ist das Ziel meines Praxisprojektes, den Lernprozess über die industriellen Testverfahren zu dokumentieren und mir einen Einblick in den Ablauf der Produktion von Leiterkarten zu geben.
    % Das obsolete Testsystem von Rohde \& Schwarz  