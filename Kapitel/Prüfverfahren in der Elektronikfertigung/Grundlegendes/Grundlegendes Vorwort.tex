\subsection{Grundlegendes}
    \label{subsection: Grundlegendes}
    Elektronische Einrichtungen kommen mittlerweile vermehrt zum Einsatz.
    Sie finden Anwendung in vielen Bereichen des Alltags mit unterschiedlichsten Anforderungen und man möchte sie kaum mehr missen.
    Von der Industrie für Unterhaltungselektronik, bis hin zur Industrie für eingebettete Steuerungssysteme, wo die Elektronik für besonders anspruchsvolle sicherheits- und zeitkritische Anwendungen, wie der Steuerung der Energieversorgung und der Steuerung der Computersysteme im Transportwesen \cite{eggersglus_test_2014}, eingesetzt wird.
    Allen ist gemeinsam, dass ihren elektronischen Erzeugnissen durch Leitungen und Verbindungen vernetzte Komponenten auf Leiterkarten zugrunde liegen.

%----------------------------- Prototypisches -----------------------------%
    % Deprecated
        % Elektronische Einrichtungen sind mittlerweile überall verbaut.
        % Sie finden Anwendung in vielen Bereichen mit unterschiedlichsten Anforderungen.
        % Von der Industrie für Unterhaltungselektronik, bis hin zur Branche für eingebettete Systeme, die einen starken Zuwachs \citneeded{Statistik} erfahren hat und die Elektronik für besonders anspruchsvolle sicherheits- und zeitkritische Anwendungen, wie der Steuerung der Energieversorgung und der Steuerung der Computersysteme im Transportwesen \cite{eggersglus_test_2014}, nutzt.
        % Sie haben gemeinsam, dass ihren elektronischen Systemen durch Leitungen und Verbindungen vernetzte Komponenten auf Leiterplatten zugrunde liegen, die in ihrem Zusammenspiel die durch den Hersteller spezifizierten Anforderungen erfüllen sollen.

        % Ermöglicht wird dieser technologische Fortschritt durch Verbesserungen in der mikroelektronischen Schaltungstechnik \cite{grout_integrated_2006}.
        % \annot{->}Ermöglicht wird dieser vermehrte Einsatz durch Verbesserungen in der mikroelektronischen Schaltungstechnik \cite{grout_integrated_2006}, denn die Komponentengrößen sind kleiner geworden, sodass Geräte kostengünstiger...
        % Es herrscht zudem ein regelrechter Bedarf an immer kleineren und kostengünstigeren Produkten mit ständig verbesserten und vereinfachten Funktionen für den Konsumenten \cite{berger_test-_2012}.

        % Daher soll untersucht werden, welchen Einfluss die Prüftechnik auf diesen technologischen Fortschritt ausübt.

        % Einen großen Zuwachs erfährt nicht nur die Branche der Unterhaltungselektronik \citneeded{Statistik}.
        % Diese elektronischen Systeme sind oft essentiell zur Ausführung sicherheits- und zeitkritischer Steuerungsaufgaben, beispielsweise in der Energieversorgung oder im Transportwesen \cite{eggersglus_test_2014}.

        % Diesen Systemen liegen die durch Leitungen und Verbindungen vernetzten Komponenten auf Leiterplatten zugrunde, die in ihrem Zusammenspiel bestimmte, durch den Hersteller festgelegte, Aufgaben erfüllen sollen.
%----------------------------- Prototypisches -----------------------------%