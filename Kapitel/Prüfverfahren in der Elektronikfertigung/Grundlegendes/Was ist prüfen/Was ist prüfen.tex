\subsubsection{Was ist prüfen?}
    \label{subsubsection: Was ist pruefen}
    Ein Produkt zeichnet sich durch seine Attribute aus, die durch Anforderungen an das Produkt festgelegt werden.
    Beispielsweise spielen die Abmessungen, die Funktion, die Masse und die Funktionsgenauigkeit, aber auch sicherheitstechnische Aspekte eine wichtige Rolle, die beim Entwurf eines Produktes festgelegt werden müssen.
    Diese Anforderungen stellen die individuellen Erwartungen der Nutzer an das Produkt dar.
    Durch die Zusammenfassung dieser Anforderungen zu einer übergeordneten Menge werden die Qualitätsmerkmale definiert, die die Eignung eines Produktes für den vorgesehenen Zweck messen.
    In der Entwicklungsphase werden diese Qualitätsmerkmale als die im Produktionsprozess zu erreichenden und zu sicherenden Sollqualitätsmerkmale, unter Absprache zwischen den Auftraggebern und den jeweiligen Prozesssteuerungsbereichen, ausgehandelt.
    Während des Produktionsprozesses können eine Vielzahl von Störgrößen auf den Herstellungsprozess einwirken, sodass sich daraus resultierend die Istqualität des Erzeugnisses von der geforderten Sollqualität unterscheidet. \cite{karger_pruftechnik_1985}
    Durch den Prüfvorgang, der nach der nachfolgenden Definition definiert ist, können diese Abweichungen ermittelt und bewertet werden:
    
    \begin{center}
        \begin{minipage}{0.75\textwidth}
            \minisec{Definition}
                Das Feststellen, ob bestimmte Eigenschaften (Merkmale) an Körpern, Stoffen, Feldern, oder Prozessen vorhanden sind bzw. ob sich ihre quantitativen Charakteristika innerhalb vorgegebener Toleranzen befinden, bezeichnet man als Prüfen. \cite{karger_pruftechnik_1985}
        \end{minipage}
    \end{center}

    Meist erfolgt die Bewertung der Abweichung anhand eines Normals, also einem Prüfgegenstand, der die vorgegebenen Qualitätsmerkmale erfüllt.
    Bei Baugruppen, die eine Menge von Leiterkarten einer Art zusammenfassen, handelt es sich bei dem Normal um ein sogenanntes Golden Board\footnote{Das Golden Board kann als ideale Version einer Leiterkarte, die als Referenz und Musterbeispiel für die \glqq Abkömmlinge\grqq\@ einer Baugruppe dient, betrachtet werden.} \cite{berger_test-_2012}.
    Als Fehler kann somit die Abwesenheit, bzw. die Abweichung eines dieser Qualitätsmerkmale vom Prüfgegenstand über den Toleranzbereich hinaus, verstanden werden.
    Messen, als unabdingbares Werkzeug der Prüftechnik, bezeichnet den Vorgang, die Qualitätsmerkmale zu quantifizieren, bzw. numerisch bestimmen zu können. \cite{karger_pruftechnik_1985}

    Die Prüfverfahren werden auch dazu genutzt, um ein Design unter verschiedensten Konditionen, die in der finalen Anwendung eines Produktes auftreten können, zu evaluieren.
    Hierzu wird das Produkt beim Prüfvorgang an den Grenzen der Betriebsspezifikationen betrieben, die vom Auftraggeber gefordert werden. \cite{grout_integrated_2006} 

    Damit die Prüfung eines Erzeugnisses überhaupt möglich ist, muss dieses die beiden Eigenschaften der Prüfbarkeit erfüllen: \cite{grout_integrated_2006}

    \begin{enumerate}
        \item \textbf{Kontrollierbarkeit}: Bestimmte Schaltungsteile eines Systems müssen manipulierbar sein, damit bestimmte Testwerte\footnote{Hierbei handelt es sich bei digitalen Schaltungen um logische Werte $\mathcal{L}_{2} \in \{0;1\}$ und bei analogen Schaltungen um Spannungen oder Ströme.} als Signale an bestimmten Testpunkten angelegt werden können.
        \item \textbf{Feststellbarkeit}: Das System muss eine messbare Antwort auf die angelegten Testwerte liefern können.
    \end{enumerate}

    Die Problematik der Prüfverfahren gestaltet sich dadurch, dass diese Prüfungsvoraussetzungen möglichst kosteneffizient umgesetzt werden müssen, denn ein Teil dieser Testkosten entfällt als Anteil auf jeden Prüfling, was seinen Preis, entgegen des Bedarfs der Endkunden, steigert \cite{grout_integrated_2006}.
    Die Tätigkeit des Prüfens ist somit Mittel der \textbf{Qualitätssicherung}, da Fehler entdeckt und beseitigt werden können \cite{karger_pruftechnik_1985}.