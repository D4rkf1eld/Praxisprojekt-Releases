\subsection{Optische Prüfverfahren}
    Die optischen Prüfverfahren haben in der Prüftechnik elektronischer Leiterkarten einen besonderen Stellenwert.
    Ihnen zugrunde liegt die kontaktlose Detektion von Fehlern durch optische Mess- oder Warnehmungsinstrumente, wie einer Kamera oder dem menschlichen Auge.
    Unterschiede liegen in der Art und Weise, wie sie ausgeführt werden und in welchem Spektralbereich des Lichts die Detektion der Fehler stattfindet.
    Somit ergeben sich leicht unterschiedliche Anwendungszwecke, die in den folgenden Abschnitten besprochen werden.
    Zu den optischen Prüfverfahren zählt die \ac{moi}, bei der die Fehlerbewertung extern durch einen menschlichen Prüfer erfolgt \cite{berger_test-_2012}, die \ac{aoi} \annot{...} und die \ac{axi}, \notsure{die auch von einem Rechnersystem gesteuert wird}.

    Der Haupteinsatzzweck dieser Prüfverfahren liegt in der Qualitätsbeurteilung sichtbarer Merkmale. 
    Das Alleinstellungsmerkmal gegenüber weiteren Prüfverfahren schlechthin ist die Bewertbarkeit von Lötstellen, speziell den Lötmenisken, zur Optimierung des Fertigungsprozesses mithilfe von Rückkopplungsschleifen.
    Dies ist möglich, da die meisten \ac{aoi} eine Statistik über die Prüfungsresultate und Qualitätsmerkmale im Prüfprozess anfertigen.
    Auch können messtechnisch nicht bestimmbare Größen, wie z.B. die Polung eines Elektrolytkondensators, durch die optische Erfassung erkannt werden. \cite{berger_test-_2012}

    Als eine weitere Besonderheit ist die Fähigkeit der \ac{axi} zu nennen, die es erlaubt, die innere Struktur einer Lötstelle zu bewerten, um somit eine voraussichtliche Lebensdauer angeben zu können, sodass weitere Optimierungsmaßnahmen getroffen werden können \cite{berger_test-_2012}.

    Weiter gestaltet sich die Handhabung der optischen Prüfverfahren in der Wertschöpfungskette innerhalb der Elektronikfertigung als flexibel.
    So lassen sich diese Prüfverfahren schon im sehr frühen Produktionsstadium als Zwischenschritte zu den normalen Herstellungsschritten ergänzen, da der Prüfling nicht funktionstüchtig sein muss. 
    Somit kann frühzeitig eine Rückmeldung über mögliche aufgetretende Fehler abgegeben werden. \cite{berger_test-_2012}
    Beispielsweise kann eine Leiterkarte nach dem Bestückungsprozess auf möglichen Bauteilversatz kontrolliert werden, bevor das Produkt durch nachfolgende Testverfahren mit Strom versorgt wird und möglicherweise beschädigt wird \cite{blunk_testverfahren_nodate}.
    Zudem wird die flexible Handhabung dadurch gestützt, dass der Prüfling zur Überprüfung nicht adaptiert werden muss.
    Somit entstehen praktisch auch keine Initialkosten, die auf die Prüflinge entfällt. \cite{berger_test-_2012}

    Da die optischen Prüfverfahren die Mängel direkt an der Leiterkarte veranschaulichen, ist die Fehlerlokalisierung zusätzlich einfacher \cite{berger_test-_2012}.

    Der Hauptgrund, warum die optischen Prüfverfahren nicht als einziges Prüfmittel und nur zur Ergänzung genutzt werden können ist, dass keine Aussage über die korrekte Funktion des Prüflings getätigt werden kann, da die Funktion der einzelnen Bauelemente nicht im Fokuspunkt steht \cite{berger_test-_2012}.

    Im Folgenden sollen nun die optischen Prüfverfahren näher beleuchtet werden.


    