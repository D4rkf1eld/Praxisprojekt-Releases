\subsubsection{Fehlerarten}
    \label{subsubsection: Fehlerarten}
    Nun folgt ein Überblick über sämtliche Fehlerarten, die beim Fertigungsprozess eines elektronischen Produktes entstehen, bzw. in diesen einfließen können.

    \minisec{Bauteilfehler}
        Unter den Bauteilfehlern gibt es rein elektrische Fehler, mechanische Mängel und mechanische Mängel, die Auswirkung auf die elektrische Funktionalität ausüben \cite{berger_test-_2012}.

        \begin{itemize}
            \item \textbf{Defektes Bauteilgehäuse:} Ein defektes Bauteilgehäuse kann sowohl für mechanische, als auch für elektrische Fehler verantwortlich sein. Die Detektion des Fehlers ist mithilfe der optischen Testverfahren (vgl. \refneeded{-> Optische Testverfahren}) möglich. Zu einem rein mechanischen Fehler zählt zum Beispiel eine abgebrochene Bauteilecke \cite{berger_test-_2012}. Zu den mechanischen Fehlern mit Auswirkung auf die elektrische Funktionalität zählen Haarisse, austretendes Elektrolyt aus einem Kondensator \cite{berger_test-_2012} und nicht benetzbare, bzw. unvollständig metallisierte Bauteilanschlüsse \cite{stiny_fertigung_2010}.
            \item \textbf{Verbogener Bauteilanschluss:} Als ein weiterer mechanisch und elektrisch wirkender Bauteilfehler ist der verbogene Bauteilanschluss zu nennen. Durch diese Fehlerart ist kein Kontakt zwischen der Leiterkarte und dem Bauteil möglich. Zusätzlich können sich die Bauteilpins so verbogen haben, dass Kurzschlüsse zwischen zwei benachbarten Pins auftreten können. Ein Sonderfall der Terminologie ergibt sich bei sogenannten Gullwing\footnote{Gullwing bezeichnet die Form der Anschlüsse und erinnert an die Form von Flügeltüren bei manchen Autos.}-Anschlüssen. Hier wird der verbogene Bauteilanschluss als \glqq lifted lead\grqq\@ bezeichnet. \cite{berger_test-_2012}
            \item \textbf{Fehlender Bauteilanschluss:} Falls der fehlende Anschluss des Bauteils der Signalübertragung dient, so wird der fehlende Bauteilanschluss als mechanischer und elektrischer Fehler klassifiziert. Wichtig ist jedoch zu beachten, dass ein fehlender Bauteilanschluss ohne elektrische Funktion, bedingt durch die interne Funktionsweise eines Chips, trotzdem für den mechanischen Halt auf der Leiterkarte wichtig ist. Ohne diesen Anschluss ist das Auftreten weiterer Fehler (z.B. der Wackelkontakt als intermittierender Fehler) durch mechanische Überbeanspruchung wahrscheinlicher.
            \item \textbf{Fehlerhafter Bauteilkennwert:} Zu den rein elektrischen Defekten zählt der falsche Bauteilkennwert. Dabei handelt es sich um einen Fehler, den das Bauteil an sich mitbringt, sodass seine Kennwerte die im Datenblatt genannten Werte bzw. Toleranzbereiche verlässt. \cite{berger_test-_2012} Bei den Bauteilkennwerten unterscheidet man zwischen den Basiskennwerten, wie zum Beispiel dem Widerstands- oder Kapazitätswert und den Nebenspezifikationen, die nicht messtechnisch bestimmbar sind. Hierzu zählen beispielsweise der Temperaturbereich oder die Spannungsfestigkeit eines Bauteils. \cite{neumann_mut_2014}
        \end{itemize}

    \minisec{Bestückfehler}
        Entsteht während des Bestückungsprozesses auf der Leiterkarte ein Fehler, so zählt dieser zu den Bestückfehlern \cite{berger_test-_2012}.

        \begin{itemize}
            \item \textbf{Fehlbestücktes Bauteil:} Weist ein Bauteil einen optisch festzustellenden Unterschied zu dem für den entsprechenden Einbauplatz vorgesehenen Bauteil auf, dazu zählt z.B. eine falsche Beschriftung, eine andere Gehäuseform, ein gänzlich anderes Bauteil oder auch die Farbe eine Leuchtdiode, so handelt es sich hierbei um ein fehlbestücktes Bauteil. Aufgrund der oftmals manuellen Bestückung der \ac{tht}-Komponenten tritt dieser Fehler dort häufig auf. \cite{berger_test-_2012} Im Falle der \ac{smd}-Bausteine tritt eine Fehlbestückung bei einer falschen \glqq Vergurtung\grqq\@ der Bestückungsmaschine oder bei einer falschen Nachrüstung auf \cite{stiny_fertigung_2010}.
            \item \textbf{Falsch positioniertes Bauteil:} Aufgrund eines falsch positionierten Bauteils entsteht ein Versatz zwischen der Soll- und Ist-Position. Dieser Versatz führt oftmals zu einem mangelnden Kontakt zu den Anschlusspads auf der Leiterkarte. Jedoch kann es auch passieren, dass dem Bauteil ein nicht mehr ausreichender Einbauplatz zur Verfügung steht. Ursache des Versatzes ist der Bestück- oder Lötprozess. Daher muss die Position des Bauteils auf der Leiterkarte erst nach dem entsprechenden Prozess überprüft werden. Dies gilt insbesondere für das \ac{smd}-Löten, da sich die Bauteile aufgrund der Oberflächenspannung des Lotes erst zentrieren und \glqq einschwimmen\grqq\@ müssen. Typische Fehlerbilder sind verdrehte Bauteile, Rückenlieger, oder Bauteile, die auf ihrer Seite liegen (Billboard). \cite{berger_test-_2012}  
            \item \textbf{Verpoltes Bauteil:} Ein verpoltes Bauteil, z.B. ein Kondensator, eine Diode oder ein \ac{ic} wurde entgegen seiner vorgesehenen Polarität aufgelötet \cite{berger_test-_2012}. Problematisch und auch teilweise gefährlich ist dies, wenn die Bauteilverpolung das Bauteil beschädigt, so z.B. als ein platzender Elektrolytkondensator, oder wenn durch diesen Fehler weitere Bauteile beschädigt werden (z.B. durch nun aus dem Kondensator austretendes Elektrolyt). Als Sonderfall müssen unipolare Bauteile, wie z.B. Widerstände, betrachtet werden \cite{berger_test-_2012}. Da die Stromrichtung bei diesen Bauelementen eigentlich egal sein sollte, tritt hier so gesehen auch kein Fehler auf. Jedoch lässt sich hieraus ein Hinweis auf einen Prozessfehler herleiten, der, zur Beibehaltung einer hohen Fertigungsqualität und Fertigungsüberwachung, behoben werden sollte \cite{berger_test-_2012}.
            \item \textbf{Fehlendes Bauteil:} An dem Einbauplatz befindet sich kein Bauteil. Der Einbauplatz ist absolut leer. \cite{berger_test-_2012}
            \item \textbf{Zu viel bestücktes Bauteil:} In der Elektronikfertigung werden häufig mehrere Iterationen, also Ausbaustufen, einer Baugruppe hergestellt. Dadurch entstehen mehrere Baugruppenvarianten, die unterschiedlich bestückt werden müssen. Dadurch müssen Einbauplätze unbestückt bleiben. Befindet sich an diesem Platz dennoch ein Bauteil, so wird dieser Fehler als ein zu viel bestücktes Bauteil klassifiziert. \cite{berger_test-_2012}
            \item \textbf{Unsachgemäße Bestückung:} Eine Unsachgemäße Bestückung liegt dann vor, wenn die für die Bestückung notwendige Vorsicht nicht geboten ist. Dies kann zum Beispiel der Fall sein, wenn \ac{mos}\footnote{\ac{mos}-Feldeffekttransistoren sind am Gate-Anschluss besonders empfindlich gegenüber statischer Aufladung.} Bauelemente bei der Bestückung aufgrund der elektrostatischen Aufladung zerstört werden \cite{karger_pruftechnik_1985}.
        \end{itemize}

    \minisec{Lötfehler}
        Die Evaluation der Lötfehler, die während des Lötprozesses auftreten, ist wichtig zur Bestimmung der Güte und zur Verbesserung des Lötprozesses. 
        Da aus Lötfehlern nicht unbedingt Kontaktfehler zwischen dem Bauteil und der Leiterkarte resultieren, die die elektrische Leitfähigkeit messbar mindern, können zum größten Teil nur optische Prüfverfahren zur Fehlerbestimmung eingesetzt werden. \cite{berger_test-_2012}

        \begin{itemize}
            \item \textbf{Kalte Lötstelle:} Die für die Testverfahren am problematischsten festzustellende Fehlerart ist die kalte Lötstelle, da bei ihr die Ausfälle nur sporadisch auftreten. Eine Detektion durch optische Prüfverfahren ist zudem nicht möglich, da der Lötmeniskus normal ausgeprägt ist. Somit unterscheiden sich die Schichtdicken der Lötstelle auch nicht vom normalen Kontakt, sodass eine Röntgeninspektion ebenfalls nicht möglich ist. Jedoch hält die kalte Lötstelle aufgrund des mangelnden physischen Kontakts zu der Leiterplatte physischen Belastungen kaum stand, sodass nur mechanische oder zerstörende Stresstests für die Detektion geeignet sind. \cite{berger_test-_2012}
            \item \textbf{Tombstone-Effekt:} Der Tombstone-Effekt (z. Dt. Grabsteineffekt) macht sich durch die Aufrichtung eines chipförmigen Bauteils während des Lötprozesses bemerkbar. Der Grund ist die unterschiedliche Wärmeaufnahmefähigket, aufgrund unterschiedlicher Geometrien, der Anschlusspads der Leiterkarten im \ac{smd}-Lötofen, sodass sich die aufgetragenden Lötpasten unterschiedlich schnell verflüssigen. Die im flüssigen Zustand zeitlich unterschiedlich aufgetretende Oberflächenspannung sorgt für das Aufrichten des Bauteils, indem an den Anschlusskappen unterschiedlich stark \glqq gezogen\grqq\@ wird. \cite{berger_test-_2012}
            \item \textbf{Benetzungsfehler:} Benetzungsfehler werden durch Verunreinigungen, z.B. durch Hautkontakt oder Oxidation, an den Bauteil- oder Leiterkartenpads hervorgerufen. Durch die unzureichende Bildung der intermetallischen Phase bildet sich die Lötstelle nicht wie gewünscht aus, sodass konvexe Lötstellen entstehen oder das Lot fließt nicht am Bauteilanschluss an. \cite{berger_test-_2012}
            \item \textbf{Unzureichende Füllhöhe:} Eine unzureichende Füllhöhe des Lotes der \ac{tht}-Durch-kontaktierung einer Leiterplatte sorgt für eine geringe mechanische Widerstandsfähigkeit und eine geringe elektrische Leitfähigkeit. Dabei spielen die Qualität der Durchkontaktierung, die Wärmeverteilung beim Lötprozess und das Verhältnis des Durchmessers des Bauteilanschlusses zum Durchmesser der Durchkontaktierung eine Rolle. Im Gegensatz zum \ac{smd}-Lötprozess, wo die Lotmenge vorher durch das Auftragen der Lotpaste auf die Leiterkarte bestimmt werden kann, kann beim \ac{tht}-Lötvorgang die von der Lötstelle aufgenommene Lotmenge nicht bestimmt werden. Somit kann es passieren, dass sich auf der Lötseite ein sehr guter Lötmeniskus bildet, jedoch der Durchstieg auf der Bauteilseite leer ist. \cite{berger_test-_2012}
            \item \textbf{Lunker:} Lunker bezeichnen Hohlräume (engl. voids) innerhalb einer Lötverbindung. Diese Hohlräume finden sich in jeder Lötstelle wieder und sind meist unkritisch, solange der Anteil der Hohlräume in der Lötverbindung nicht einen bestimmten Prozentsatz übersteigt. Einen Einfluss auf die Menge und Größe der Lunker hat das Alter, die Verschmutzung und der Typ der verwendete Lötpaste. Vermeiden kann man Lunker durch das Löten im Vakuum, da die Luft aus dem flüssigen Lot in Folge des Unterdrucks entweichen kann. In Folge der Lunker kann z.B. die Wärmeabfuhr eines Bauteils nicht mehr gewährleistet werden. \cite{berger_test-_2012}
            \item \textbf{Black-Pad:} Das Black-Pad ist im klassischen Sinn kein Lötfehler, sondern ein dunkles chemisches Reaktionsprodukt aus Nickeloxid ohne Leitfähigkeit und mit geringer Tragfähigkeit, das schon in der Leiterplattenfertigung entsteht. Statt der für das Anschlusspad gewünschten Nickel-Zinn Phase, entsteht, durch die Störung des Veredlungsvorganges mit Nickel-Gold, zwischen dem Lot und Lötpad ein dunkles Nickeloxid-Pad. Das Black-Pad ist durch elektrische Prüfverfahren nachweisbar, wenn das komplette Lötpad ein Black-Pad ist. Zu einem normalen Anschlusspad bestehen keine signifikanten Unterschiede, sodass die optische Prüfverfahren hier nicht eingesetzt werden können. Da das Black-Pad aufgrund der geringen Haftung unter dem Lot sofort zum Vorschein kommt, ist eine Prüfung durch zerstörende Tests möglich. \cite{berger_test-_2012}
            \item \textbf{Kurzschluss:} Ein Kurzschluss entsteht durch Lötbrücken bei zu viel Lot, wobei der Abstand zwischen zwei benachbarten Lötstellen und die Qualität des Aufdrucks vom Lötstopplack auf der Leiterplatte entscheidend ist und als Lötkugel zwischen zwei Bauteilanschlüssen, oder bei Verunreinigungen durch elektrisch leitfähige Metallspäne oder andere Fremdkörper. Bei den \ac{tht}-Bauelementen spielt die letzendlich zu verwendende Lotmenge eine wichtige Rolle, da eine richtige Dosierung des Lotes vorab nicht möglich ist. Teilweise kann ein Kurzschluss zwischen zwei Pins keine Auswirkungen auf die Funktionsweise haben, wenn beide Anschlüssen auf dem selben elektrischen Potential liegen. Unter Umständen sollte nicht immer die kleinste Bauteilform gewählt werden, da dies, aufgrund der geringeren Abstände zwischen den Bauteilanschlüssen, die Wahrscheinlichkeit eines Kurzschlusses erhöht. \cite{berger_test-_2012}
            \item \textbf{Unzureichende Lötverbindung:} Die unzureichende Lötverbindung kennzeichnet alle Fehler, die die bereits besprochenen Fehlerarten nicht abdecken. Die notwendige Bedingung ist jedoch, dass trotzdem eine elektrische Verbindung besteht. Bei dieser Fehlerart weisen die Anschlüsse zu viel Lot, zu wenig Lot, einen mangelhaften Lötmeniskus, oder Lötperlen auf. Bestimmende Parameter, die diese Fehler hervorrufen können, sind zum einen die Lottemperatur und zum anderen die Menge des Lotes. Im Falle der \ac{smd}-Bausteine wird die Qualität der Lötverbindungen durch weitere externe Faktoren bestimmt. Beispielsweise spielt das Aussehen der Leiterkarte eine Rolle, da es die Anordnung und Maße der Löcher für die Schablone, die der Auftragung der Lotpaste dient, bestimmt. Daher neigen bestimmte Layouts zur Verstopfung der Löcher oder zur Verschmierung der Lotpaste. \cite{berger_test-_2012}
        \end{itemize}