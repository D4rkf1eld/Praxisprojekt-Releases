\subsubsection{Motivation}
    Elektronische Einrichtungen kommen mittlerweile vermehrt zum Einsatz.
    Sie finden Anwendung in vielen Bereichen des Alltags mit unterschiedlichsten Anforderungen und man möchte sie kaum mehr missen.
    Von der Industrie für Unterhaltungselektronik, bis hin zur Industrie für eingebettete Steuerungssysteme, wo die Elektronik für besonders anspruchsvolle sicherheits- und zeitkritische Anwendungen, wie der Steuerung der Energieversorgung und der Steuerung der Computersysteme im Transportwesen \cite{eggersglus_test_2014}, eingesetzt wird.
    Allen ist gemeinsam, dass ihren elektronischen Erzeugnissen durch Leitungen und Verbindungen vernetzte Komponenten auf Leiterkarten zugrunde liegen.
    Diese intensive Nutzung elektronischer Geräte im Alltag ist überhaupt erst durch Verbesserungen in der mikroelektronischen Schaltungstechnik möglich geworden \cite{grout_integrated_2006}.
    Gleichzeitig steigt der Bedarf an immer kleineren und kostengünstigeren Produkten mit ständig verbesserten und vereinfachten Funktionen für den Konsumenten \cite{berger_test-_2012}.
    Jedoch erwarten Nutzer elektronischer Produkte eine stets hohe Zuverlässigkeit, bei stetigen Produktverbesserungen der elektronischen Geräte.
    Gerade dann, wenn diese Produkte in anspruchsvollen sicherheits- und zeitkritischen Anwendungen integriert sind, wäre ein Fehlverhalten, oft hervorgerufen durch mangelnde Qualitätskontrolle, fatal, da im schlimmsten Fall Menschenleben gefährdet werden würden.
    Daher obliegt die Feststellung der Zuverlässigkeit als ein Aspekt, der die Qualität eines Produktes ausmacht, bei der Fertigung des elektronischen Erzeugnisses, bzw. muss in den einzelnen Produktionsphasen bescheinigt werden.
    Für zuverlässige Systeme stellen die Testverfahren somit ein unerlässliches Mittel der Sicherheitsüberprüfung und Qualitätssicherung dar, wobei gleichzeitig der Kostenaspekt mit berücksichtigt werden muss.
    Deswegen gilt es zu überprüfen, wie die Testverfahren auf die folgenden Aspekte einwirken:

    \begin{itemize}
        \item Qualitätssicherung
        \item Kostenminimierung
        \item Sicherheit
    \end{itemize}

    Diese Auswirkungen werde ich in Abschnitt \refneeded{-> Notwendigkeit der Testverfahren} näher untersuchen.