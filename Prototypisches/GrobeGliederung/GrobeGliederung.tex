\setcounter{section}{0}

\section{Einleitung}
    Dieses Praxisprojekt entsteht in der Zusammenarbeit mit der Trützschler Group SE in Mönchengladbach Odenkirchen\dots

    \subsection{Die Trützschler Group}

        \annot{Unternehmen vorstellen}

        \annot{Was macht Trützschler?}

        \annot{Was ist das besondere bei Trützschler?}

    \subsection{Problemstellung}

        \annot{TSAS ist alt}

            \annot{Keine Ersatzteile mehr für TSAS}
            
            \noindent
            \annot{Software}

                \annot{Keine Updates mehr}

                \annot{Es ist schwierig neue Leiterkarten dort einzubinden}

                \annot{Neues System nutzt bestehende CAD-Dateien}

            \noindent
            \annot{Adapterbau ist teuer und aufwendig}

                \annot{Beispiel anbringen für den Preis und den Aufwand (Euro und Stunden)}

                \annot{Debugging der Adapter ist auch aufwendig}
                
                \annot{Daher sollen alte Adapter auf dem neuen System funktionieren}

    \subsection{Ziele des Praxisprojektes}

        \annot{Mir einen Einblick in den Ablauf der industriellen Produktion von Leiterkarten geben}

        \noindent
        \annot{Über verschiedene Testmöglichkeiten informieren}

        \noindent
        \annot{R\&S und SPEA dokumentieren}

        \noindent
        \annot{Maschinen kennenlernen}

            \annot{Gemeinsamkeiten und Unterschiede der Tester erfahren}

            \annot{Mögliche Migrationsstrategien stellen sich dabei heraus}

    \subsection{Aufbau des Praxisprojektes}

\section{Testverfahren in der Industrie}

    \subsection{Herstellungsschritte der Leiterkarten}

        \annot{Unbestückte Leiterkarte/PCB heißt dann PCBA (Printed Circuit Board Assembly)}

        \noindent
        \annot{Von der Planung bis zur fertigen Karte}

            \annot{\notsure{Vielleicht als Flussdiagramm?}}

            \annot{Bestückungsreihenfolge}

                \indent \indent \annot{Hinsichtlich der Bauelemente}

            \annot{Verschiedene Testverfahren, die durchlaufen werden}

                \indent \indent \annot{Von dem ICT mit Kurzschlusstest (sehr am Anfang) bis hin zur Wärmekammer}

                    \indent \indent \indent \annot{\notsure{Vielleicht als Flussdiagramm?}}

        \noindent
        \annot{\notsure{Wie designed man for Test?}}

            \annot{Regeln aufführen}

            \annot{Unter dem Aspekt, dass einige Leiterkarten nicht ICT fähig sind}

                \indent \indent \annot{Hohe Bauteile}

                \indent \indent \annot{Bottom Bestücktheit}

                \indent \indent \annot{Wie wird mit solchen Boards verfahren? \notsure{Flying Probe?}}

    \subsection{Notwendigkeit der Testverfahren}

        \annot{Fehler die auftreten können (Fehlerarten/Fehlertypen) und dann sagen, dass das Testen sinnvoll ist}

            \annot{Z.B. kann ein Widerstand falsch sein/Bauteilwert falsch sein}

                \indent \indent \annot{Würde man erstmal so nicht drauf kommen}

            \annot{Bei BGA ist ein Pin nicht richtig verlötet}

                \indent \indent \annot{Würde man erstmal so nicht drauf kommen}

        \noindent
        \annot{Statistik ansprechen, wieviele Leiterkarten gebaut werden}

            \annot{Sagen wieviele davon durchfallen}

            \noindent
            \annot{Grund ist die Qualitätssicherung, Kostenminimierung und Sicherheit für die Inbetriebnahme}

                \indent \annot{Durch das Testen können außerdem Verbesserungen an dem Produkt vorgenommen werden}

                \indent \annot{Das ganze mit Minisections gliedern}

    \subsection{Testverfahren}

        \subsubsection{Standard In Circuit Testing \dbg{<- Acronym}}

            \annot{Manufacturing Defect Analysis (MDA)}

            \noindent
            \annot{Design-for-Testability}
                \notsure{Nochmals anschneiden und spezifische Werte füllen?}

            \noindent
            \annot{Adapter}

                \annot{Single Bay oder Dual Bay}

                \annot{Nadelarten}

            \noindent
            \annot{Single Core und Multi Core Systeme}

        \subsubsection{Flying Probe Testing}

        \subsubsection{BIST}

        \subsubsection{Optische Testverfahren}

            \annot{Man kann die Polarität von Elkos nicht messtechnisch bestimmen, daher ist das hier sinnvoll!}

            \noindent
            \annot{Manuelle Optische Inspektion}

            \noindent
            \annot{Automatische Optische Inspektion}

        \subsubsection{Röntgeninspektion}

        \subsubsection{\notsure{Funktionstest}}

        \subsubsection{\notsure{Zusätzliche Features wie ISP}}

        \subsubsection{weitere Recherche nötig\dots \annot{Weitere und gucken ob es auch VERFAHREN sind!}}

    \subsection{Teststrategien}

        \subsubsection{Kapazitäten \annot{Wie kann man die Kapazität/andere Bauteile messen}}

            \annot{Dieser Teil hier soll unabhängig des Testverfahrens sein}

                \annot{\notsure{Also ist der Boundary Scan zu den oberen Punkten (ICT, FPT, usw.) zugehörig?}}

        \subsubsection{Analoger Test}

            \annot{\notsure{Vectorless?}}
            
            \noindent
            \annot{\notsure{Hier Open Pin Test als Alternative für den Digitaltest anführen?}}

                \indent \annot{\notsure{Digitaltest benötigt ja hohe Schaltfrequenzen und perfektes Timing}}

                \indent \annot{\notsure{Werden feste Testmuster (bekannte Ein- und Ausgabe) benötigt?}}

        \subsubsection{Digitaler Test}

            \annot{\notsure{Vector Test?}}

            \noindent
            \annot{\notsure{Bei Geräte die keinen Boundary Scan unterstützen?}}

        \subsubsection{Mixed Signal Test}

        \subsubsection{Boundary Scan (JTAG)}

            \annot{Boundary Scan Descriptive Language}

        \subsubsection{Junction Scan/Open Pin Test}

        \subsubsection{Funktionstest}

    \subsection{Maßnahmen zur Messfehlerreduzierung}

        \subsubsection{Guarding}

            \annot{Laut T.W. gibt es hierfür mehrere Möglichkeiten}

        \subsubsection{Mehrdrahtmessung}

        \subsubsection{weitere Recherche nötig\dots}

\section{Praxisbeispiele}

    \subsection{Rohde \& Schwarz TSAS}
        \subsubsection{Grundlegendes}

        \subsubsection{Aufbau/Funktionsweise/Schnittstellen}

        \subsubsection{Module/Technische Spezifikationen}

        \subsubsection{Besonderheiten}

        \subsubsection{Probleme}

    \subsection{SPEA 3030 Compact}
        \subsubsection{Grundlegendes}
        
        \subsubsection{Aufbau/Funktionsweise/Schnittstellen}

            \annot{Augat Pylon Interface}

        \subsubsection{Module/Technische Spezifikationen}

        \subsubsection{Besonderheiten \annot{Stray Capacitance Ausgleich z.B.}}

        \subsubsection{Probleme}

    \subsection{Kompatibles}

        \subsubsection{Vorschläge, wie die Migration erfolgen könnte (in der BA prüfen und sich entscheiden?)}

    \subsection{Unterschiede/Inkompatibilitäten}

        \subsubsection{Wie könnte man damit verfahren (in der BA ansprechen und einen Weg raussuchen?)}

\section{Zusammenfassung}

\section{Ausblick}

    \noindent
    \annot{Ausblick über allgemeine Testverfahren}

    \noindent
    \annot{\notsure{Ausblick über die Verfahrenswege der Migration}}

\section{Fazit}