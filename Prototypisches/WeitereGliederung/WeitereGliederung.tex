% \setcounter{section}{0}

\section*{Einleitung}

    \subsection*{Vorwort}

    \subsection*{Problemstellung}

    \subsection*{Ziele des Praxisprojektes}

        \annotdone{Mir einen Einblick in den Ablauf der industriellen Produktion von Leiterkarten geben}

        \noindent
        \annotdone{Über verschiedene Testmöglichkeiten informieren}

        \noindent
        \annotdone{R\&S und SPEA dokumentieren}

        \noindent
        \annotdone{Maschinen kennenlernen}

            \annotdone{Gemeinsamkeiten und Unterschiede der Tester erfahren}

            \annotdone{Mögliche Migrationsstrategien stellen sich dabei heraus}

    \subsection*{Aufbau des Praxisprojektes}

\section*{Die Trützschler Group}

    \subsection*{Allgemeines}

        \annotdone{Jetzigen Text mit etwas geschichtlichem Hintergrund füllen}

    \subsection*{Geschäftszweige}

        \annot{Mit Bildern füllen}

    \subsection*{Besonderheiten}

        \annotdone{Fast alles wird selber hergestellt}

        \annotdone{Hochqualitative Leiterkartenproduktion in DEUTSCHLAND}

\section{Prüfverfahren in der Elektronikfertigung}

    \annot{Kriterien für die Testverfahren siehe blunk-electronic oder Kärger}

    \subsection*{Notwendigkeit der Prüfverfahren}

        \annot{Statistik ansprechen, wieviele Leiterkarten gebaut werden}

            \annot{Sagen wieviele davon durchfallen}

        \noindent
        \annotdone{Grund ist die Qualitätssicherung, Kostenminimierung und Sicherheit für die Inbetriebnahme}

            \indent \annotdone{Durch das Testen können außerdem Verbesserungen an dem Produkt vorgenommen werden}

            \indent \annotdone{Zum Punkt Kostenminimierung: Rule Of Ten oder Kärger die $10^{s}$ Formel.}

            \indent \annotdone{Das ganze mit Minisections gliedern}

    \subsection*{Fehler}

        \annotdone{Fehler die auftreten können (Fehlerarten/Fehlertypen) und dann sagen, dass das Testen sinnvoll ist}

            \annotdone{Z.B. kann ein Widerstand falsch sein/Bauteilwert falsch sein}

                \indent \indent \annotdone{Würde man erstmal so nicht drauf kommen}

        \annotdone{Bei BGA ist ein Pin nicht richtig verlötet}

            \indent \indent \annotdone{Würde man erstmal so nicht drauf kommen}

    \subsection{Phasen des Herstellungsprozesses}

        \annot{\notsure{Stiny Buch hier nützlich?}}

        \annot{\notsure{Sicherheitskritische Tests, z.B. Schutz vor Berührungen von 230V garantieren; Hierzu Normen o.ä. finden}}

        \noindent
        \annot{Wann muss/sollte getestet werden}

            \annot{Hierzu die Präsentation von blunk-electronic!}
        
        \noindent
        \annot{Unbestückte Leiterkarte/PCB heißt dann PCBA (Printed Circuit Board Assembly)}

        \noindent
        \annot{Von der Planung bis zur fertigen Karte}

            \annot{\notsure{Vielleicht als Flussdiagramm?}}

            \annot{Bestückungsreihenfolge}

                \indent \indent \annot{Hinsichtlich der Bauelemente}

            \annot{Verschiedene Testverfahren, die durchlaufen werden}

                \indent \indent \annot{Von dem ICT mit Kurzschlusstest (sehr am Anfang) bis hin zur Wärmekammer}

                    \indent \indent \indent \annot{\notsure{Vielleicht als Flussdiagramm?}}

        \subsubsection{Entwicklungsphase}

            \annot{Design For Test anführen}

            \annot{Warum ist Design For Test so wichtig?}

            \noindent
            \annot{Wie designed man for Test?}

            \noindent
            \annot{Grundlegende Regeln aufführen}

                \annot{Weitere und speziellere Regeln bei dem jeweiligen Verfahren nennen}

            \noindent
            \annot{Unter dem Aspekt, dass einige Leiterkarten nicht ICT fähig sind}

                \indent \indent \annot{Hohe Bauteile}

                \indent \indent \annot{Bottom Bestücktheit}

                \indent \indent \annot{Wie wird mit solchen Boards verfahren? \notsure{Flying Probe?}}

        \subsubsection{Produktionsphase}

        \subsubsection{Verifikationsphase}

            \annot{Testen der Leiterkarten}

    \subsection{Optische Prüfverfahren}

        \annot{Man kann die Polarität von Elkos nicht messtechnisch bestimmen, daher ist das hier sinnvoll!}

        \subsubsection{Manuelle Optische Inspektion}

            \annot{Design For Test}

        \subsubsection{Automatische Optische Inspektion}

            \annot{Design For Test}

        \subsubsection{Röntgeninspektion}

            \annot{Design For Test}

    \subsection{Elektrische Prüfverfahren}

        \subsubsection{In Circuit Testing}

            \annot{Manufacturing Defect Analysis (MDA)}

            \annot{Prüfparameter der Bauelemente \refneeded{-> (Kärger/K4SK1)}}

            \noindent
            \annot{Design-for-Testability}
                \notsure{Nochmals anschneiden und spezifische Werte füllen?}

            \noindent
            \annot{Adapter}

                \annot{Single Bay oder Dual Bay}

                \annot{Nadelarten}

            \noindent
            \annot{Single Core und Multi Core Systeme}

            \annot{Design For Test}

        \subsubsection{Flying Probe Testing}

            \annot{Design For Test}

        \subsubsection{Funktionstest}

            \annot{Eigentlich ist hierfür doch ein OBP (On Board Programming) mit \notsure{ISP?} nötig?}

            \noindent
            \annot{Design For Test}

        \subsubsection{Build in self Test}

            \annot{Eigentlich ist hierfür doch ein OBP (On Board Programming) mit \notsure{ISP?} nötig?}

            \noindent
            \annot{Design For Test}

    \subsection{Messstrategien}

        \subsubsection{Analoger Test}

            \annot{Wie kann man die Kapazität/andere Bauteile messen}

            \noindent
            \annot{\notsure{Vectorless?}}

            \noindent
            \annot{\notsure{Hier Open Pin Test als Alternative für den Digitaltest anführen?}}

                \indent \annot{\notsure{Digitaltest benötigt ja hohe Schaltfrequenzen und perfektes Timing}}

                \indent \annot{\notsure{Werden feste Testmuster (bekannte Ein- und Ausgabe) benötigt?}}
        
            \noindent
            \annot{Maßnahmen zur Messfehlerreduzierung}

                \annot{Guarding}

                    \annot{Laut T.W. gibt es hierfür mehrere Möglichkeiten}

                \annot{Vierdrahtmessung/Kelvinmessung}

                \annot{\notsure{Stromrichtige Messung}}

                \annot{\notsure{Spannungsrichtige Messung}}

        \subsubsection{Digitaler Test}

            \annot{\notsure{Vector Test?}}

            \noindent
            \annot{\notsure{Bei Geräte die keinen Boundary Scan unterstützen?}}

        \subsubsection{Mixed Signal Testing}

        \subsubsection{Junction Scan/Open Pin Scan}

        \subsubsection{Boundary Scan (JTAG)}

            \annot{Verschiedene IEEE Standards nennen von .1 bis .4 analog boundary scan}
            
            \noindent
            \annot{Boundary Scan Descriptive Language}

\section{Praxisbeispiel: Rohde \& Schwarz TSAS}

    \subsection{Grundlegende Informationen}

    \subsection{Aufbau}

        \annot{Augat Pylon Interface}

    \subsection{Funktionsweise}

    \subsection{Technische Spezifikationen}

        \subsubsection{Module}

        \subsubsection{Schnittstellen}

    \subsection{Besondere Merkmale}

    \subsection{Probleme}

\section{Praxisbeispiel: SPEA 3030 Compact}

    \subsection{Grundlegende Informationen}

    \subsection{Aufbau}

        \annot{Augat Pylon Interface}

    \subsection{Funktionsweise}

    \subsection{Technische Spezifikationen}

        \subsubsection{Module}

        \subsubsection{Schnittstellen}

    \subsection{Besondere Merkmale}

        \annot{Stray Capacitance Ausgleich z.B.}

    \subsection{Probleme}

\section{Kompatibilität und mögliche Verfahrenswege}

    \annot{Vorschläge, wie die Migration erfolgen könnte (in der BA prüfen und sich entscheiden?)}

    \noindent
    \annot{Inkompatibles}

        \annot{Wie könnte man damit verfahren (in der BA ansprechen und einen Weg raussuchen?)}

\section{Zusammenfassung}

\section{Ausblick}

    \noindent
    \annot{Ausblick über allgemeine Testverfahren}

    \noindent
    \annot{\notsure{Ausblick über die Verfahrenswege der Migration}}

\section{Fazit}